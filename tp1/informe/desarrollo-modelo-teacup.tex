% TODO REVISAR
\subsubsection{Modelo gráfico}
Como se puede observar en la figura \ref{fig:Teacup_sd} el modelo cuenta con 1 (un) stock para modelar la temperatura de la taza, que llamamos \textit{Teacup Temperature}, 2 (dos) auxiliares, que en este caso son constantes, pero podrían ser funciones, una para la temperatura del cuarto, (\textit{Room Temperature}, y otra para el tiempo característico, \textit{Characteristic Time}), y un flujo de salida (outflow) para la perdida de calor de la taza hacia el cuarto,\textit{Heat Loss to Room}, con origen el stock \textit{Teacup Temperature} y destino vacío. 

Las flechas negras indican que el flujo de salida proveniente de \textit{Teacup Temperature} utiliza dicha constantes auxiliares en su función interna para determinar el valor de dicho flujo de salida en cada instante en el tiempo. La flecha azul, indica que dicha función también utiliza el valor del stock \textit{Teacup Temperature} para hacer este cálculo. 

\begin{figure}[!h]
\centering
\includegraphics[scale=0.5]{imagenes/Teacup_sd.jpg}
\caption{Modelo Teacup expresado en System Dynamics en formato gráfico}
\label{fig:Teacup_sd}
\end{figure}

\begin{figure}[!h]
\centering
\includegraphics[scale=0.5]{imagenes/teacup_mapeo/Teacup_variables}
\caption{Modelo Teacup exprsado en System Dynamics en formato XMILE}
\label{fig:Teacup_xmile}
\end{figure}

Como se puede observar en la figura \ref{fig:Teacup_xmile}, se utilizan distintos tags para los elementos de System Dynamics, uno para los flujos de output e inflow (el tag \textbf{flow} en el archivo xmile), otro para las constantes auxiliares (el tag \textbf{aux}) y otro para los stocks (el tag \textbf{stock}). 

Asimismo, en cada flujo, se utiliza un tag \textbf{eqn} en archivo xmile para mostrar la función utilizada por dicho \textit{flow} (que puede ser tanto de input ó output) para agregarle o quitarle unidades respectivamente al stock sobre el que operan. 

Observamos que en los stocks y variables auxiliares se utiliza el tag \textbf{eqn} para mostrar el valor inicial de dicho stock ó variable auxiliar. En este caso, las variables auxiliares son todas constantes, con lo cuál en el tag \textbf{eqn} contiene el valor que se mantendrá igual durante toda la simulación.

A partir de estas observaciones, decidimos representar este mismo modelo en el formalismo DEVS, de la forma en que se muestra en el diagrama de la figura \ref{fig:Teacup_devs_flattened}

\begin{figure}[!h]
\centering
\includegraphics[scale=0.5]{imagenes/Teacup_devs_flattened}
\caption{Modelo Teacup expresado en DEVS en formato gráfico}
\label{fig:Teacup_devs_flattened}
\end{figure}

En el diagrama de la figura \ref{fig:Teacup_devs_flattened} se puede observar lo siguiente: el \textbf{stock} se corresponde en el modelo DEVS con dos atómicos.

\begin{itemize}
	\item Un integrador QSS1 de nombre \textit{"Teacup Temperature Integrator"} (\texttt{utilizamo el nombre del stock más la palabra Integrator})
	\item Un atómico Ftot que determina la variación de unidades que tendrá dicho stock, utilizando todos los inflows y outflows que operan sobre dicho stock ($\sum inflows - \sum outflows $). Los inflows se conectarán al puerto \texttt{inPlus} y los outflows al \texttt{inMinus}. 
\end{itemize}

Por otro lado, a cada flujo (en este caso \textit{Heat Loss to Room} se corresponde con dos atómicos, uno representando el flujo de entrada (inflow) y otro la flujo de salida (outflow). Esto es para darle mejor legibilidad al modelo.

Decidimos nombrarlo de esta manera, en el caso de un \textit{outflow} concatenando \textit{Fminus} el nombre del \textit{stock} sobre el cual opera y el nombre del flujo, en el caso del demolo teacup \textit{Fminus Teacup Temperature Heat Loss to Room}.

En el caso de un inflow, el nombre es igual salvo que reemplazamos \textit{Fminus} por \textit{Fplus}. En el ejemplo de la figura \ref{fig:Teacup_devs_flattened}, el atómico que representa el inflow \textit{Heat Loss to Room} de no cumple ninguna función en el modelo DEVS y es por ello que lo marcamos en rojo. 

Finalmente, a los auxiliares (tag \textbf{aux}) del modelo SD se corresponde con atómicos en el modelo DEVS. En este caso, los atómicos emitirán un valor constante a través de su puerto out. Dado que los valores de los auxiliares \textit{Room Temperature} y \textit{Characteristic Time} es utilizado por el flujo \textit{Heat Loss to Room} para el calculo del valor de la función en el modelo SD, en el modelo DEVS estos atómicos se conectarán con los correspondientes a los atómicos del flujo (no realizamos las conexiones al atómico correspondiente al inflow ya que no es utilizado en el modelo).

También podemos observar en el diagrama del modelo DEVS una línea azul la cual se corresponde con la línea azul del diagrama \ref{fig:Teacup_sd} (en el modelo SD esto es la utilización de \textit{Teacup Temperature} en el calculo de \textit{Heat Loss to Room}). De esta forma, mostramos que el atómico \textit{FminusTeacupTemperature - Heat Loss to Room} utiliza también el output del atómico \textit{Teacup Temperature Integrator} para realizar sus cálculos. 

%TODO Falta revisar para abajo
Llegados a esta instancia, nos preguntamos qué pasaría con nuestro mapeo a DEVS si más de un outflow operara sobre \textit{Teacup Temperature}. Así, decidimos pensar cómo sería la traducción gráfica del siguiente modelo (figura \ref{fig:Teacup_sd_2}), que modela un segundo outflow que opera sobre el stock \textit{Teacup Temperature} (en el ejemplo, decimos que la taza no sólo se enfría de acuerdo al rate de pérdida de calor de la taza contra la habitación, sino también contra una máquina enfriadora \textit{Cooling Machine} que agregamos - como podría ser por ejemplo un ventilador -). Llegamos a la conclusión que el modelo DEVS correspondiente a este nuevo modelo debería ser el de la figura \ref{fig:Teacup_devs_flattened_2}. 

Como se puede ver, agregamos el atómico \textit{FminusTeacupTemperature - Heat Loss to Cooling Machine}, con los mismos inputs que \textit{FminsuTeacupTemperature - Heat Loss to Room}, pero con la novedad de que los outputs de estos atómicos llegan a distintos puertos de entrada del atómico Ftot que teníamos antes. Estos puertos de entrada están nombrados de acuerdo al nombre del flujo que opera sobre el stock correspondiente al Ftot en cuestión (en este caso, \textit{Heat Loss to Room} y \textit{Heat Loss to Cooling Machine} ambos operan sobre \textit{Teacup Temperature} en carácter de quitadores de unidades), para que el modelo gráfico que más claro.

\begin{figure}[!h]
\centering
\includegraphics[scale=0.4]{imagenes/Teacup_sd_2}
\caption{Modelo Teacup (versión 2) expresado en SD en formato gráfico}
\label{fig:Teacup_sd_2}
\end{figure}

\begin{figure}[!h]
\centering
\includegraphics[scale=0.4]{imagenes/Teacup_devs_flattened_2}
\caption{Modelo Teacup (versión 2) expresado en DEVS en formato gráfico}
\label{fig:Teacup_devs_flattened_2}
\end{figure}

Finalmente, pensamos en una forma de modularizar el modelo acoplado tan complejo que quedó tras las conversiones, tanto en la versión original como en la versión 2 del modelo Teacup. Lo que obtuvimos, fue lo que se muestra en las figuras \ref{fig:Teacup_devs} y \ref{fig:Teacup_devs_2}.

Aquí se puede observar que definimos un modelo acoplado por cada \textbf{stock}, que engloba al integrador correspondiente, a la Ftot correspondiente, a los Fminus's correspondientes (outflows que tienen como origen al \textbf{stock} en cuestión), y a los Fplus's correspondientes (inflows que contienen como origen al \textbf{stock} en cuestión). Vale la pena aclarar aquí que dichos, inflows y outflows que operan sobre cada \textbf{stock} corresponden cada uno de ellos a un \textbf{flow} en el modelo expresado en System Dynamics, y es interesante notar que visualmente es muy simple e intuitivo ver cómo se relacionan cada uno de los modelos expresados en SD con su contraparte en DEVS. 

Todos estos serán factores muy importantes a tener en cuenta a la hora de desarrollar la primera versión del traductor, ya que querremos que esta sea lo más general posible, y que sea fácilmente modificable para ser capaz de representar nuevos modelos cada vez más complejos a medida que el proyecto avance.

\begin{figure}[!h]
\centering
\includegraphics[scale=0.35]{imagenes/Teacup_devs}
\caption{Modelo Teacup expresado en DEVS en formato gráfico utilizando varios niveles de acoplamiento}
\label{fig:Teacup_devs}
\end{figure}
\begin{figure}[!h]
\centering
\includegraphics[scale=0.35]{imagenes/Teacup_devs_2}
\caption{Modelo Teacup (versión 2) expresado en DEVS en formato gráfico utilizando varios niveles de acoplamiento}
\label{fig:Teacup_devs_2}
\end{figure}

\subsubsection{Modelo en código}
En esta sección, primeramente mostraremos el modelo .ma que deseamos ser capaces de generar a partir del archivo .devsml generado por nuestro traductor a partir del archivo .xmile del modelo Teacup que ya mencionamos previamente. Por razones de tiempo y para simplificar las cosas en esta primera aproximación al problema, sólo trabajaremos con la versión aplanada del modelo. Es decir, no intentaremos generar varias capas de acoplados, sino un sólo acoplado Top, que contendrá a todos los atómicos que hagan falta.

\begin{figure}[!h]
\centering
\includegraphics[scale=0.5]{imagenes/teacup_mapeo/Teacup_ma}
\caption{Archivo .ma correspondiente al modelo Teacup para simular el modelo en el simulador CD++}
\label{fig:Teacup_ma}
\end{figure}

Como se puede observar en la figura \ref{fig:Teacup_ma}, el archivo .ma consta de la definición del componente acoplado Top mediante [top], la definición de todos los componentes (modelos atómicos o acoplados, aunque en este ejemplo, como dijimos van a ser todos atómicos) que forman parte del acoplado Top, los puertos de entrada y salida del acoplado (en este caso, la entrada son las variables constantes auxiliares, y la salida es la úncia variable de interés del modelo - la temperatura de la taza -) y los links entre cada uno de los componentes del acoplado.

Compárese la figura \ref{fig:Teacup_devs_flattened} con la figura \ref{fig:Teacup_ma}, y observése el mapeo 1-1 que hay entre cada línea del archivo .ma con el diagrama correspondiente.

Ahora bien, los atómicos utilizados en este modelo, deberán tener un comportamiento. Obviamente, queremos también automatizar la generación de archivos que le den comportamiento a estos atómicos, para que el modelo .ma mostrado más arriba pueda ser ejecutado. Exponemos a continuación las partes más relevantes del código C++ que nuestro traductor genera junto con el .ma para darle el comportamiento deseado a cada atómico.

\begin{figure}[!h]
\centering     %%% not \center
\subfigure[Archvo .h]{\label{fig:Teacup_h}\includegraphics[scale=0.26]{imagenes/teacup_mapeo/Teacup_h}}
\subfigure[Archivo .cpp]{\label{fig:Teacup_hpp}\includegraphics[scale=0.26]{imagenes/teacup_mapeo/Teacup_cpp}}
\caption{Código relevante de los archivos .h y .cpp generados para el atómico FminusTeacupTemperature}
\end{figure}

\begin{figure}[!h]
\centering     %%% not \center
\subfigure[Archvo Ftot.cpp]{\label{fig:Teacup_h}\includegraphics[scale=0.3]{imagenes/gral_mapeo/ftot_cpp}}
\subfigure[Archivo Cte.cpp]{\label{fig:Teacup_hpp}\includegraphics[scale=0.3]{imagenes/gral_mapeo/cte_cpp}}
\caption{Código relevante de los archivos .cpp generados para los atómicos Cte y Ftot}
\end{figure}

\subsubsection{Generación de modelo ejecutable en CD++}
Para poder generar los archivos .ma, .h y .cpp que mencionamos en la sección anterior, previamente debemos generar un archivo .devsml que contenga toda la información necesaria para este fin.

Una vez generado dicho archivo, resta generar una descripción del modelo en un lenguaje que nuestro simulador (en este caso $CD++$), pueda entender para así ejecutar. 

Como ya vimos, se puede hacer un mapeo $1-1$ entre cada línea del archivo $DEVSML$ y del archivo $MA$. De esta forma, se completa el pipeline de transformaciones necesarias para pasar de un modelo ejecutable especificado en un formalismo de tiempo discreto hacia otro formalismo, de tiempo continuo y eventos discretos.

\begin{figure}[!h]
\centering     %%% not \center
\subfigure[Atómicos 1-1 para cada Flow]{\label{fig:Teacup_devsml_components}\includegraphics[scale=0.4]{imagenes/teacup_mapeo/Teacup_devsml_components}}
\subfigure[Atómicos 1-1 para cada Stock]{\label{fig:Teacup_devsml_stocks}\includegraphics[scale=0.4]{imagenes/teacup_mapeo/Teacup_devsml_stocks}}
\caption{Parte relevante del código .devsml generado por cada Constante, Stock y Flow}
\end{figure}

\begin{figure}[!h]
\centering     %%% not \center
\subfigure[Conexiones internas]{\label{fig:Teacup_devsml_internal_connections}\includegraphics[scale=0.4]{imagenes/teacup_mapeo/Teacup_devsml_internal_connections}}
\subfigure[Conexiones externas]{\label{fig:Teacup_devsml_external_connections}\includegraphics[scale=0.4]{imagenes/teacup_mapeo/Teacup_devsml_external_connections}}
\subfigure[Puertos de entrada/salida del modelo Top]{\label{fig:Teacup_devsml_ports}\includegraphics[scale=0.4]{imagenes/teacup_mapeo/Teacup_devsml_ports}}
\caption{Parte relevante del código .devsml generado por cada conexión y para los puertos del modelo Top}
\end{figure}

% TODO
\subsection{Modelo Formal}
Primero mostramos formalmente los modelos atómicos utilizados en el acoplado Top, con la descripción de su comportamiento interno.
\begin{itemize}

\item \textbf{Cte} : $ roomTemperature, characteristicTime \rightarrow \langle X, S, Y, \delta_{int}, \delta_{ext}, \lambda, t_{a} \rangle$ \newline
\begin{itemize}
	\item $ X = \{ inValue \} $ \newline
	\item $ S = \{ value \} $ \newline
	\item $ Y = \{ out \} $ \newline
	\item $ \delta_{int}(\langle value \rangle) = \emptyset $ \newline
	\item $ \delta_{ext} (\langle value \rangle, e, x)= \{ value := x.value \} $ \newline
	\item $ \lambda(\langle value \rangle, out) = value $ \newline
	\item $ t_{a}(s) = \infty $ 
\end{itemize}

\item \textbf{Fminus} : $ fmTeacupTemperatureHeatLossToRoom \rightarrow \langle X, S, Y, \delta_{int}, \delta_{ext}, \lambda, t_{a} \rangle$ \newline
\begin{itemize}
	\item $ X = \{ inRoomTemperature, inCharacteristicTime, inTeacupTemperatureIntegrator \} $ \newline
	\item $ S = \{ roomTemperature, characteristicTime, teacupTemperatureIntegrator, isSetRoomTemperature, \newline isSetCharacteristicTime, isSetTeacupTemperatureIntegrator \} $ \newline
	\item $ Y = \{ out \} $ \newline
	\item $ \delta_{int}(s) = \emptyset $ \newline
	\item $ \delta_{ext}(s, e, x) = \{
	\\if (x.port = inRoomTemperatureroomTemperature) roomTemperature := x.value; isSetRoomTemperature := true
	\\if (x.port = inCharacteristicTime) characteristicTime := x.value; isSetCharacteristicTime := true
	\\if (x.port = inTeacupTemperatureIntegrator) teacupTemperatureIntegrator = x.value; \\isSetTeacupTemperatureIntegrator := true 
	\} $ \newline
	\item $ \lambda(s, out) = if(todas \ las \ variables \ seteadas) \{ 
\\(teacupTemperatureIntegrator-roomTemperature)/characteristicTime\} \ else \ \emptyset$ \newline
	\item $ t_{a} = \infty $ 
\end{itemize}

\item \textbf{Ftot} : $ ftTeacupTemperature \rightarrow \langle X, S, Y, \delta_{int}, \delta_{ext}, \lambda, t_{a} \rangle$ \newline
\begin{itemize}
	\item $ X = \{ inMinusHeatLossToRoom \} $ \newline
	\item $ S = \{ plus, minus \} $ \newline
	\item $ Y = \{ out \} $ \newline
	\item $ \delta_{int}(\langle plus, minus \rangle) = \emptyset $ \newline
	\item $ \delta_{ext}(\langle plus, minus \rangle, e, x) = \{ 
	\\if (x.port = inPlus) plus := x.value
	\\if (x.port == inMinusHeatLossToRoom) minus := x.value
	\} $ \newline
	\item $ \lambda(\langle plus, minus \rangle, out) = (plus - minus) $ \newline
	\item $ t_{a} = \infty $ 
\end{itemize}
\item \textbf{QSS1} : $ teacupTemperatureIntegrator \rightarrow \langle X, S, Y, \delta_{int}, \delta_{ext}, \lambda, t_{a} \rangle$ \newline
\begin{itemize}
	\item $ X = \{ in \} $ \newline
	\item $ S = \{ ? \} $ \newline
	\item $ Y = \{ out \} $ \newline
	\item $ \delta_{int}(s) = \{ ? \} $ \newline
	\item $ \delta_{ext}(s, e, x) = \{ ? \} $ \newline
	\item $ \lambda(s) = ? $ \newline
	\item $ t_{a} = \{ ? \} $ 
\end{itemize}
\end{itemize}

Ahora que ya tenemos lo atómicos, expresamos el acoplado que utiliza a los atómicos expuestos más arriba:
\begin{itemize}
\item $ Top \rightarrow \langle X, Y, \{ M_{1}, M_{2}, M_{3}, M_{4}, M_{5} \}, C_{xx}, C_{yx}, C_{yy}, Select \rangle$ \newline
\begin{itemize}
	\item $ X = \{ \} $ \newline
	\item $ Y = \{ \} $ \newline
	\item $ M_{1} = roomTemperature $ \newline
	\item $ M_{2} = characteristicTime $ \newline
	\item $ M_{3} = fmTeacupTemperatureHeatLossToRoom$ \newline
	\item $ M_{4} = ftTeacuptTemperature $ \newline
	\item $ M_{5} = teacupTemperatureIntegrator $ \newline
	\item $ C_{xx} = $ \{ \} \newline
	\item $ C_{yx} = \{ (M_{1}.!out, M_{3}.?inRoomTemperture), (M_{2}.!out, M_{3}.?inCharacteristicTime), \\
(M_{5}.!out, M_{3}.?inTecupTemperatureIntegrator), (M_{3}.!out, M_{4}.?inMinusHeatLossToRoom), \\
(M_{4}.!out, M_{5}.?in) \} $ \newline
	\item $ C_{yy} = \{ \} $ \newline
\end{itemize}
\end{itemize}
