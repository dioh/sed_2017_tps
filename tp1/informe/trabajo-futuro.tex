Algunas cosas que nos han quedado afuera de este trabajo que podrían ser incorporadas en futuros trabajos que extiendan este.

\begin{itemize}
	\item Traducción de modelos conocidos más complejos
	\item Incorporar características de los modelos \textit{System Dynamics} descritos por el estándar que han quedado afuera del presente trabajo. Por ej:
	\begin{itemize}
		\item Auxiliares con funciones como valor de \texttt{eqn}
		\item Aceptar el uso de las funciones integradas (\textit{built-in functions})
		\item Parámetros de los \textit{stock} como \texttt{non\_negative}, \texttt{queue} o \texttt{conveyor}
		\item Funciones gráficas (\textit{Graphical functions}) 
		\item Módulos y submódulos
	\end{itemize}
	\item Traducir el stock y los flujos asociados como un acoplado
	\item Reutilización de los modelos atómicos, por ejemplo, levantando las funciones dinámicamente mediante muParser\cite{muparser} 
	\item Utilización de integradores como \texttt{QSS2} o \texttt{QSS3}
\end{itemize}