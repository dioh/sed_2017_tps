
El formalismo System Dynamics (de ahora en más SD) fue creado en 1961 por J. W.
Forrester como un mecanismo para modelar sistemas continuos. El método tuvo un
gran éxito en varias aplicaciones como por ejemplo la simulación de sistemas
socioeconómicos y de gestión. La metodología fue aplicada en un sinfín de
escenarios muy diversos: en la simulación de portfolios de negocios (Merten,
Löfler, Weidman, 1987), construcción de autopistas en China (Xu, Mashayeki,
Saeed, 1998), proceso de desarrollo de productos (Ford, Sterman, 1998),
modelización de HIV (Sterman, 2001), etc.


Se han desarrollado una gran variedad de programas de simulación basados en este formalismo. Algunos ejemplos son: Dynamo, Stella, Vensim, Simile, Simulink, Sybase, AnyLogic, PowerSim, etc. Hasta hace poco tiempo, todos estos programas eran absolutamente incompatibles, ya que no había un estándar para representar modelos (es decir, dado un mismo modelo para cierto sistema, para correrlo en los distintos programas era necesario reescribir el modelo para cada caso particular). Sin embargo, en 2013 la IEEE en conjunto con IBM establecieron un estándar para representar modelos del formalismo SD en XML denominado XMILE. El estándar permite que modelos desarrollados en diferentes programas puedan ser exportados a XMILE para luego ser importados y simulados en otros programas sin la necesidad de una total o parcial reescritura.

Por otro lado, debemos mencionar al formalismo de modelización DEVS (Discrete Event Specification System - Theory of Modeling and Simulation, Zeigler, 1976 -). Este formalismo resulta ser mucho más flexible y de mejor calidad que SD: entre otras ventajas, permite expresar no sólo sistemas continuos, sino también sistemas discretos, permite la reutilización de modelos, lanzar eventos internos o externos en el modelo, y mejora notablemente la performance en sistemas grandes.

Al igual que en el caso de SD, hay varios programas que implementan el formalismo DEVS. Podemos destacar a PowerDEVS, CD++, DEVSJAVA, Python DEVS,  DEVSim++, etc. Con este formalismo sucedía exactamente lo mismo que con SD, hasta que se creó un estándar denominado DEVSML (ver ‘Mittal, José Luis Risco-Martín, Zeigler , DEVSML: Automating DEVS Execution Over SOA Towards Transparent Simulators’).

Si bien es posible transformar un modelo en System Dynamics a DEVS por medio de transformaciones intermedias, no existe ninguna transformación o programa que haga una traducción directa de un modelo a otro. 

Buscaremos en este trabajo de investigación responder las siguientes preguntas teóricas.
¿Es posible realizar dicha transformación? Creemos que sí y que la implementación de un traductor que permita pasar entre estos dos formalismos sería sumamente útil ya que permitiría compatibilizar y combinar modelos desarrollados en SD con modelos desarrollados en DEVS de forma cómoda y sin la necesidad de efectuar reescrituras de modelos.
Y no menos importante, ¿existe un método general que nos permita validar y testear que dicha transformación es válida?

La inspiración para este trabajo proviene de observar el hecho que System Dynamics es un formalismo que expresa modelos de forma gráfica. En DEVS, los modelos tanto atómicos como acoplados también pueden ser expresados de forma gráfica. Entonces, la idea será encontrar un mapeo uno a uno (en la medida de lo posible) de modelos SD y modelos DEVS. A continuación, mostramos un ejemplo de un modelo SIR, y su posible mapeo de stocks, flows y feedbacks con modelos atómicos DEVS. Por supuesto, esto es sólo la parte gráfica. Una parte importante del trabajo será definir de qué forma definir el comportamiento de cada modelo atómico en función del comportamiento de los respectivos stocks, flows y feedbacks.

Para cumplir con este objetivo, se implementará un traductor que transformará un modelo SD expresado en XML en el estándar para el formalismo SD (XMILE) en un modelo DEVS expresado en XML en el estándar para el formalismo DEVS (DEVSML).
Desarrollaremos técnicas estadísticas y de visualización para poder validar la salida de los modelos transformados.

La utilización de este traductor facilitará la reutilización de modelos ya construidos y estudiados, y permitirá su modificación y adaptación a problemas de la actualidad, aprovechando todas las ventajas del formalismo DEVS.

Asimismo, los modeladores especialistas en SD podrán transformar sus modelos de forma automática a DEVS, sin tener que aprender los detalles de dicho formalismo. Esto ayudará en la integración de modelos de tiempo continuo desarrollados en System Dynamics con modelos de eventos discretos desarrollados en DEVS.



