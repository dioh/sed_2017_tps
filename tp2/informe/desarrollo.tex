Para poder simular el comportamiento de los Shockers fue necesario extender el trabajo en distintos puntos explicados a continuación.

\subsection{Modelo atómico Shocker}

El comportamiento exógeno fue modelado con $K$  modelos atómicos. Cada uno de éstos modelos afecta a una porción de las celdas totales de la grilla.
En particular, si asumimos una grilla cuadrada de $N \times N$ y $K$ Shockers, cada uno afectará $N^2/K$ celdas.

De las celdas a las cuales el Shocker puede influir, solo una porción recibirá un shock. Las celdas afectadas por cada Shocker se decide con probabilidad uniforme. No todas las celdas serán afectadas.

Los Shockers emitirán en momentos específicos valores específicos. Cuando sus funciones de salida ejecuten, algunas celdas recibirán estos valores y actuarán de acuerdo a cierta lógica descrita en la siguiente sección.


\subsection{Nuevas reglas para la grilla}

Para poder estudiar la influencia de los Shockers determinamos las siguientes reglas para la grilla.

\begin{itemize}
    \item rule: {  uniform(-3, -2) } 0 { (0,0,0) = 5 }
    \item rule: {  uniform(-1, 1)  } 0 { (0,0,0) = 6 }
    \item rule: {  uniform(2, 3) } 0 { (0,0,0) = 7 } 
\end{itemize}

Para sintetizar el comportamiento agregado podemos decir que existirán tres tipos de Shockers. Los que emitan el valor 5 polarizarán a las celdas hacia el intervalo [-3, -2]. Los que emitan el valor 6, neutralizarán la opinión de estas celdas volviéndolas indecisas. Por otra parte al emitirse el valor 7, las celdas afectadas polarizarán hacia el intervalo opuesto [2, 3].


\subsection{Experimentos}

Diseñamos experimentos para responder algunas preguntas, como por ejemplo:
\begin{itemize}
\item ¿Cómo afecta la presencia o no de Shockers al estado final del sistema (partidismo vs. bipartidismo)$?$ 
\item ¿Con qué frecuencia se deberían generar shocks en la población para que estos tengan un efecto sobre la dinámica de intercambio de opinión de la misma?
\item ¿Cuál es el factor más determinante del estado final de la simulación: el estado inicial de las celdas o la cantidad / características / frecuencia de los Shockers?
\end{itemize}

Definimos:

\begin{description}
    \item \textbf{Tamaño de la grilla}  $10 \times 10$
    \item \textbf{Cantidad de Shockers} 5.
    \item \textbf{Tiempo entre shocks} 10 segundos. Todos los Shockers actuarán en simultaneo.
    \item \textbf{Tiempo de influencia entre vecinos} 100 milisegundos. Se espera mayor interacción entre vecinos que shocks.
    \item \textbf{Tiempo virtual de simulación} 4 minutos.
    \item \textbf{Distribución poblacional inicial} 10\% de indecisos, 45\% de votantes de A, 45\% votantes de B.
\end{description}

Las variaciones entre los experimentos están dadas por los efectos por cada Shocker.

Distingamos los tres tipos de shocks que pueden observarse en nuestro modelo.
Los shocks de tipo A, llevan a una celda a un valor con distribución
uniforme [-3, -2]. Los shocks de tipo I, neutralizan
llevando las celdas al intervalo  [-1, 1]. Por último los shocks de tipo B
polarizan hacia los valores [2, 3]. Siempre con distribución uniforme.

Este diseño experimental pretende reproducir casos donde los influenciadores
llevan adelante una competencia territorial,eventualmente llegando a una
situación donde uno prevalece logrando votación mayoritaria a su favor

\subsection{Escenarios} % (fold)
\label{sub:Escenarios}

% subsection Escenarios (end)
\begin{itemize}
    \item \textbf{Shockers: 5A} Todos los Shockers obran a favor del partido A.
    \item \textbf{Shockers: 2A 2I 1B} Hay mayor dominación del partido A, sin embargo existen 2 Shockers que generan indecisos.
    \item \textbf{Shockers: 2A 1I 2B} Terreno neutral, ninguno de los dos partidos domina.
    \item \textbf{Shockers: 2A 3B} Hay dominación por parte del partido B, pero ligeramente.
    \item \textbf{Shockers: 1A 3I 1B} Terreno neutral con mayor cantidad de Shockers que generan indecisos.
    \item \textbf{Shockers: 1A 4B} La mayor cantidad de Shockers dominan a favor del partido B pero el partido A tiene un socker que genera opinión a su favor.  
\end{itemize}

