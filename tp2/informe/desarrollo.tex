Para poder simular el comportamiento de los shockers fue necesario extender el trabajo en distintos puntos.

\subsection{Modelo atómico Shocker}

El comportamiento exógeno fue modelado con n modelos atómicos. Cada uno de éstos modelos afecta a una porción de las celdas totales de la grilla.
En particular, si asumimos una grilla cuadrada de $N\ times N$ y $K$ Shockers, cada uno afectará $N^2/K$ celdas.

Las celdas afectadas por cada Shocker se decide con probabilidad uniforme. No todas las celdas serán afectadas.

Los shockers emitirán en momentos específicos y valores específicos. Cuando sus funciones de salida ejecuten, algunas celdas recibirán estos valores y actuarán de acuerdo a cierta lógica descrita en la siguiente sección.


\subsection{Nuevas reglas para la grilla}

Para poder estudiar la influencia de los shockers determinamos las siguientes reglas para la grilla.

\begin{itemize}
    \item rule: {  uniform(-3, -2) } 0 { (0,0,0) = 5 }
    \item rule: {  uniform(-1, 1)  } 0 { (0,0,0) = 6 }
    \item rule: {  uniform(2, 3) } 0 { (0,0,0) = 7 } 
\end{itemize}

Para sintetizar el comportamiento agregado podemos decir que existirán tres tipos de shockers. Los que emitan el valor 5 polarizarán a las celdas hacia el intervalo [-3, -2]. Los que emitan el valor 6, neutralizarán la opinión de estas celdas volviendolas indecisas. Por otra parte al emitirse el valor 7, las celdas afectadas polarizarán hacia el intervalo opuesto [2, 3].


\subsection{Experimentos}

Para analizar el comportamiento del modelo diseñamos pruebas para poder analizar el efecto de los shocks variando el porcentaje de indecisos en la población.

Definimos:

\begin{description}
    \item[Tamaño de la grilla]  $10 \times 10$
    \item[Cantidad de shockers] 5
    \item[Tiempo entre shocks] 10 segundos
    \item[Tiempo virtual de simulación] 4 minutos
    \item[Distribución poblacional inicial] 10\% de indecisos, 45\% de votantes de A, 45\% votantes de B.
\end{description}

Las variaciones entre los experimentos están dadas por los efectos por cada shocker.

Supongamos el vector de influencias [X, X, X, X, X]. En cada posición tenemos la influencia que puede generar el shocker de la posición i-ésima. Los valores pueden ser \textit{A, B o I}. Es decir, partido A, partido B o indeciso. Generamos los siguientes vectores de influencias que se mantienen a lo largo de toda la simulación.

Detrás del diseño de estos experimentos se esconde la idea de competencia por territorios. En donde el que mayor influencia tiene en él, puede dominar la partida o generar una votación a su favor.

\begin{description}
    \item[Shockers: 5A] Todos los shockers obran a favor del partido A.
    \item[Shockers: 2A 2I 1B] Hay mayor dominación del partido A, sin embargo existen 2 shockers que generan indecisos.
    \item[Shockers: 2A 1I 2B] Terreno neutral, ninguno de los dos partidos domina.
    \item[Shockers: 2A 3B] Hay dominación por parte del partido B, pero ligeramente.
    \item[Shockers: 1A 3I 1B]Terreno neutral con mayor cantidad de shockers que generan indecisos.
    \item[Shockers: 1A 4B]La mayor cantidad de shockers dominan a favor del partido B pero el partido A tiene un socker que genera opinión a su favor.
    
\end{description}

