En los procesos de intercambio de opinión, las interacciones interpersonales influyen en la toma de decisiones, tienen amplia aplicación en estudios de comportamiento de las personas como por ejemplo en las ciencias sociales, políticas y económicas. Dichos procesos, son de gran interés para el estudio de procesos electorales en los cuales la opinión pública tiene directa injerencia en la decisión del electorado.

En este trabajo, continuaremos el estudio del impacto de los indecisos y su
interacción con otros actores de opinión más fuerte y su consecuente impacto en
el resultado de una elección bi-partidista. Extenderemos este trabajo
enriqueciendo el comportamiento de los agentes, es decir modelos con la capacidad de tener una opinión e implementados como autómatas celulares en cell-devs, modelando su capacidad de reaccionar ante eventos externos,
que se disparan independientemente del estado de la dinámica de intercambio de
opinión entre los agentes afectados. A estos eventos los
llamaremos \textbf{shocks de opinión}, que pueden pensarse como eventos que
afectan un conjunto de agentes e influyen su opinión de manera exógena.

Entonces, a partir de la realización de esta extensión al modelo original, pretendemos ser capaces de modelar y analizar nuevos aspectos de la realidad, como por ejemplo el impacto de la influencia de los medios de información y las redes sociales en los indecisos del electorado promedio.

El objeto de estudio utilizado  es de una elección entre dos partidos (por ejemplo, en el ámbito de un ballotage presidencial), sometida a un libre intercambio de opiniones. Nos basamos en el trabajo de Dina y Sosa\cite{dina2015}  que a su vez es una implementación del trabajo de Balenzuela et.al\cite{balenzuela}.
