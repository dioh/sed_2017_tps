
Observamos que la interacción de Shockers sobre celdas con mucha influencia
partidaria afecta de manera muy fuerte a los resultados finales.
En los casos donde la influencia partidaria se da a favor de uno de ellos, la
polarización se observa claramente y se puede observar un resultado
unipartidista.


Nuestra propuesta de agregar shocks externos enriqueció la dinámica de
intercambio de opinión aunque por los parámetros elegidos esta dinámica de
interacción entre celdas fue opacada por la manipulación de opinión dada por
los shocks exógenos.


A modo de trabajo futuro sería importante extender los experimentos en
distintos ejes. Para poder entender cómo afectan los shocks a las celdas. Por
ejemplo disminuyendo la proporción de celdas afectadas en cada shock; variando la 
intensidad de los shocks pero a intervalos mas pequeños;
cambiar la distribución de valores iniciales; probar con tiempos entre shocks
más largos; variar la cantidad de shockers y su balanceo.


