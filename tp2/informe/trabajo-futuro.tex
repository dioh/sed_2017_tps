
Observamos que la interacción de shockers sobre las celdas afecta fuertemente la opinión
final de estas. 
En los casos donde la influencia de los shockers es mayoritariamente a uno de los valores (A, B o I), la
tendencia hacia dicho valor se observa claramente, independientemente de los valores finales observados sin la influencia de shocker.

Creemos que nuestra propuesta de agregar shocks externos enriqueció la dinámica de
intercambio de opinión y puede servir para generar nuevos modelos de estudio, por ejemplo donde los shocks representen aspectos de la economía.

A modo de trabajo futuro sería importante extender los experimentos en
distintos ejes para poder entender cómo afectan los shocks a las celdas. Por
ejemplo: disminuyendo la proporción de celdas afectadas en cada shock; variando la 
intensidad de los shocks pero a intervalos mas pequeños;
cambiar la distribución de valores iniciales; probar con tiempos entre shocks
más largos; variar la cantidad de shockers y su balanceo.


